%        File: recipes.tex
%     Created: Sun Feb 26 06:00 PM 2012 P
% Last Change: Sun Feb 26 06:00 PM 2012 P
%
\documentclass[a4paper]{book}
\usepackage[index]{cuisine}
\title{Kitchen Science\\with}
\author{Molly Raven\\\&\\Zachary Bush}
\begin{document}
\maketitle
\tableofcontents
\chapter{Discoveries}
\begin{recipe}{Accidental Stroganoff}{2 Servings}{10 Minutes}
\freeform 
\ing[4]{oz}{Rotini}
Cook the rotini to your desired amount.

\ing[\fr18]{cup}{Butter}
Melt the butter in medium pot, and add in the rotini. 

\ing[\fr12]{Can}{Canned Pinto Beans}
\ing[\fr12]{Can}{Canned Black Beans}
\ing[\fr14]{cup}{Peas}
Pour the beans, including the juice, into the pot. Add in the peas as well. 

\ing[2]{Slices}{Your favorite cheese (We used gouda)}
Melt the cheese into the mixture. Stir, and then cover.

\ing[]{}{Garlic Powder}
\ing[]{}{Pepper}
\ing[]{}{Curry Powder}
Add to taste.
\end{recipe}
\chapter{Bratwurst}

\chapter{Sauces}
\begin{recipe}{Bananafister's Surprise}{About 3 Cups}{15 Minutes}
\freeform This is a very simple asian inspired sauce that goes well with many 
things. It goes great with rice, and can even be used as a salad dressing. 
\ing[1]{Tbsp.}{Olive Oil}
\ing[1]{Tbsp.}{Garlic}
\ing[1]{Tbsp.}{Ginger}
Heat up the olive oil in a medium saucepan, until warm and runny. Add in the
garlic and the ginger, be careful not to let it burn.
\ing[1]{Cup}{Soy Sauce}
\ing[1]{Cup}{Water}
\ing[\fr34]{Cup}{Brown Sugar}
Add in the water and soy sauce. Stir and heat the mixture until just boiling.
Then add in the brown sugar and heat until sauce begins to thicken. 
\ing[2]{Tbsp.}{Corn Starch}
\ing[1]{}{Banana}
While the sauce is heating, mash the banana until smooth. Combine the cornstarch
with a small amount of water, and make sure it is fully disolved. Now add the
cornstarch mixture and the banana paste. Cook until desired consistency is
reached.
\end{recipe}
\chapter{Desserts}
\begin{recipe}{Love Pebbles}{20 Servings}{10 Minutes}
\freeform Make rice crispy treats, but instead of rice crispies, use fruity
pebbles.

\end{recipe}
\end{document}
